%!TEX root = adaptive_dict_kaf.tex

\begin{abstract}
Una de las formas de estudiar y medir las propiedades estadísticas de un sistema disipativo llevado fuera del equilibrio mediante un forzamiento externo es utilizar análogos eléctricos [Fauve and Heslot, 1983; van Zon, 2004; Falcón and Falcon, 2009; Gomez-Solano, 2010; Seifert, 2012]. 
Las extremas condiciones de temperatura, radiación y disipación de energía sitúan al espacio como un ambiente hostil para estos sistemas forzados fuera del equilibrio. Siendo que gran parte o toda la tecnología utilizada para controlar los artefactos espaciales son de naturaleza electrónica, se propone estudiar la influencia del ambiente espacial sobre diferentes sistemas electrónicos disipativos simples. A modo de ejemplo, se han medido sistemas electrónicos en tierra como el sistema de control AURIGA [Bonaldi, 2009], donde se muestra un comportamiento fuera del equilibrio frente a un ambiente de temperaturas extremadamente bajas y grandes fluctuaciones de potencia. 
En este contexto que la plataforma CubeSat sirve como una alternativa natural para realizar mediciones y estudios del comportamiento del espacio sobre sistemas disipativos, pues a diferencia de otras misiones satelitales clásicas tipo NASA, las misiones CubeSat son de más simple y rápida integración de experimentos que permitan interactuar directamente con el ambiente espacial.
A raíz de lo anterior, parte de los experimentos a bordo de los satélites SUCHAI son sistemas disipativos excitados por una fuentes estocásticas con el objeto de estudiar el ambiente hostil y la termodinámica fuera del equilibrio. En particular para el SUCHAI 1 se utiliza un circuito RC en serie con una fuente de voltaje pseudo-aleatoria con espectro de frecuencia configurable [Diaz, 2016]. En esta implementación, se utiliza un circuito con un tiempo característico tau=1.21ms (R= 1.2kOhm, C=1uF) y una distribución uniforme a la entrada del circuito entre -0.8V y 0.8V de semilla conocida. El espectro de frecuencia de la señal de entrada es configurable vía tele-comandos desde 10 Hz - 6 KHz y las medidas son realizadas al voltaje del condensador shunt. 
A la fecha se tienen los datos medidos en tierra por el mismo experimento y se está a la espera de descargar todos los datos medidos en espacio para poder hacer una comparación. Este mismo tipo de experimento con circuitos electrónicos analógicos serán replicados en SUCHAI 2 y SUCHAI 3 para obtener una mejor resolución estadística del fenómeno, pues la reducida memoria de SUCHAI 1 restringe la cantidad de muestras tomadas por el ADC a un número máximo de 4000 muestras (<5 min), lo cual puede no ser suficiente en su duración como para observar algún fenómeno durante los recorridos de su órbita. Se añadirán comportamientos no-lineales vía transistores a los circuitos (SUCHAI 2) y se reducirán en tamaño hasta llegar a la escala nanométrica (SUCHAI 3) para investigar los efectos de los ambientes hostiles.

\end{abstract}
